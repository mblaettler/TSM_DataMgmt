
\section{Arrays}

\begin{breakbox}
\boxtitle{Create Table With Array:}
\sql{sql_code/array_create_table.sql}

%\begin{lstlisting}[language=SQL,aboveskip=1pt, belowskip=2pt]
%SELECT ARRAY[1,2,3+4];
%\end{lstlisting}
\end{breakbox}

\begin{breakbox}
\boxtitle{Insert:}
\sql{sql_code/array_insert.sql}
\end{breakbox}

\begin{breakbox}
\boxtitle{Constructors:}
\sql{sql_code/array_constructors.sql}
\end{breakbox}

\begin{breakbox}
\boxtitle{Select:}
\sql{sql_code/array_select.sql}
\end{breakbox}

\begin{breakbox}
\boxtitle{Operators:}
\sql{sql_code/array_operators.sql}
\end{breakbox}

\begin{breakbox}
\boxtitle{Functions:}
\sql{sql_code/array_functions.sql}

Further: not equal <>, comparison < =< > >= (lexicographical comparison of elements), contains @>, array-to-array concatenation ||.
\end{breakbox}

\begin{breakbox}
\boxtitle{CTE:}

Temporary, named result set. Can be referenced later and has recursion unlike a nested statement.

\sql{sql_code/cte_non_recursive.sql}

\sql{sql_code/cte_recursive.sql}
\end{breakbox}