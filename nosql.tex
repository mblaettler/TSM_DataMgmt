
\section{NoSQL Databases}

\begin{breakbox}
\boxtitle{Scaling:}
\begin{itemize}
	\item Scale vertically (scale up):
		\begin{itemize}
			\item add resources to single node in a system, typically involving the addition of CPUs or memory to single computer.
		\end{itemize}
	\item Scale horizontally (scale out):
		\begin{itemize}
			\item add more nodes to a system, such as adding new computer to distributed software application.
		\end{itemize}
\end{itemize}
\end{breakbox}

\begin{breakbox}
\boxtitle{CAP Theorem:}
\begin{itemize}
	\item \textcolor{Emerald}{C}osistency: The data is in every replication on every server the same.
	\item \textcolor{Emerald}{A}vailability: The data must be always available (accessible).
	\item \textcolor{Emerald}{P}artition Tolerance: The db works fine despite network and machine failures.
\end{itemize}
One can only pick two of those properties at a time.
\end{breakbox}

\begin{breakbox}
\boxtitle{Consistency:}
\begin{itemize}
	\item Strong consistency:
		\begin{itemize}
			\item after an update is committed, each subsequent access will return the updated value.
		\end{itemize}
	\item Weak consistency:
		\begin{itemize}
			\item the systems does not guarantee that subsequent accesses will return the updated value.
			\item a number of conditions might need to be met before the updated value is returned.
			\item inconsistency window: period between update and the point in time when every access is guaranteed to return the updated value.
		\end{itemize}
	\item Eventual consistency:
		\begin{itemize}
			\item Specific form of weak consistency.
			\item If no new updates are made, eventually all accesses will return the last updated values.
			\item In the absence of failures, the maximum size of the inconsistency window can be determined based on communication delays, system load and number of replicas.
			\item Domain Name System (DNS) uses eventual consistency for updates.
			\item RDBMS use eventual consistency for asynchronous replication or backup.
		\end{itemize}
\end{itemize}
\end{breakbox}

\begin{breakbox}
\boxtitle{BASE:}
\begin{itemize}
	\item \textcolor{Emerald}{B}asically \textcolor{Emerald}{A}vailable: permanent availablity
	\item \textcolor{Emerald}{S}oft State: consistency is no solid state
	\item \textcolor{Emerald}{E}ventual Consistency: data is consistent sometime
\end{itemize}
\end{breakbox}

\begin{breakbox}
\boxtitle{NoSQL Characteristics:}
\begin{enumerate}
	\item Horizontal scalable (shared nothing)
	\item Can Replicate and distribute (partition) data
	\item Have simple API (no SQL binding)
	\item Have weaker concurrency / transaction model than ACID (eventually consistent)
	\item Are schema free, have weak schema restrictions (can add new attribues at runtime)
	\item Use distributed indexes and use RAM for data storage efficiently
\end{enumerate}
\end{breakbox}

\begin{breakbox}
\boxtitle{NoSQL Categories:}
\begin{enumerate}
	\item Key/Value databases
	\item Document stores
	\item Column oriented databases
	\item Graph databases
	\item Others
\end{enumerate}
\end{breakbox}

\begin{breakbox}
\boxtitle{Key/Value Databases:}
\end{breakbox}




















