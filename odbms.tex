
\section{ODBMS}

\begin{breakbox}
\boxtitle{Manifesto:}
\newline Object-oriented systems
\begin{enumerate}
	\item Complex objects
	\item Object identity
	\item Encapsulation
	\item Types and classes
	\item Type and class hierarchies
	\item Overriding, -Loading, late binding
	\item Computational completeness
	\item Extensibility
\end{enumerate}
Database management systems
\begin{enumerate}
	\setcounter{enumi}{8}
	\item Persistence
	\item Efficiency
	\item Concurrency
	\item Reliability
	\item Declarative query language
\end{enumerate}
\end{breakbox}

\begin{breakbox}
\boxtitle{Advantages RDBMS / ODBMS:}
\newline Pro RDBMS
\begin{itemize}
	\item Neutral form of data, different views
	\item Ad-hoc queries (joins), even between non-PK fields
	\item OLAP / Cubes
	\item Cluster architectures
	\item More DB services / tools, inc. administration
	\item More sensitive to inheritance (no overlapping/disjoint in UML)
	\item Concurrency / Isolation (transactions)
\end{itemize}
Pro ODBMS
\begin{itemize}
	\item No mapping
	\item Performance (no intrinsic mapping loss; pointer based)
	\item Deep object graphs
	\item Automated schema changes
	\item No administration
\end{itemize}
\end{breakbox}

\begin{breakbox}
\boxtitle{ODBMS Products:}
\begin{itemize}
	\item Caché, db4o (C\#, Java), Versant (C++, Java), ObjectivityDB (C++), ObjectStore (C++)
\end{itemize}
\end{breakbox}

\begin{breakbox}
\boxtitle{db4o Features:}
\begin{itemize}
	\item No modification on classes
	\item Simple one liner to store objects and collections
	\item Local and Client/Server
	\item ACID transaction model
	\item Object caching, integration with Garbage Collector
	\item Automated schema evolution
	\item Java und .NET Language Binding
	\item Small “Memory foot-print” (one library 500Kb)
	\item Faster than MySQL, OR Mapper (50x?)
\end{itemize}
\end{breakbox}

\begin{breakbox}
\boxtitle{db4o Object Container:}
\begin{itemize}
	\item Represents db4o DB.
	\item Manages a transaction
		\begin{itemize}
			\item All operations in one (nested) transaction.
			\item Transaction starts implicitly when object container is opened.
			\item After commit/rollback autom. next transaction.
		\end{itemize}
	\item Manages references to stores and instantiated (transient) objects (incl. object identity).
	\item Life cycle:
		\begin{itemize}
			\item Object container open as long as programm runs.
			\item When Object container is closed, references to objects in memory are thrown away.
		\end{itemize}
\end{itemize}
\end{breakbox}

\begin{breakbox}
\boxtitle{db4o Object Container Code:}
\java{java_code/object_container.java}
\end{breakbox}

\begin{breakbox}
\boxtitle{db4o Operations:}
\begin{itemize}
	\item Store: \lstinline[language=Java]{database.store(object);}
		\begin{itemize}
			\item Object graph is being traversed and all referenced objects are made persistent.
			\item Stored whole object (deep copy).
			\item Persistence by reachability!
		\end{itemize}
	\item Retrieve: query by example, native query, SODA query (see below).
	\item Update: again use \lstinline[language=Java]{database.store(object);}
		\begin{itemize}
			\item Default for update depth is 1 (= shallow copy).
			\item Thus only basic types and Strings are updated.
			\item Object graph not being traversed (performance!).
			\item Cascading update can be configured per class.
			\item \lstinline[language=Java]{Db4o.configure().objectClass(Person.class).cascadeOnUpdate(true);}
		\end{itemize}
	\item Delete: \lstinline[language=Java]{database.delete(object);}
		\begin{itemize}
			\item Delete does not cascade by default.
			\item Referenced objects need to be deleted by hand.
			\item Cascading delete can be configured per class.
			\item \lstinline[language=Java]{Db4o.configure().objectClass(Author.class).cascadeOnDelete(true);}
		\end{itemize}
\end{itemize}
\end{breakbox}

\begin{breakbox}
\boxtitle{db4o Update object:}
\java{java_code/update.java}
\end{breakbox}

\begin{breakbox}
\boxtitle{db4o Query By Example:}
\java{java_code/query_by_example.java}
\end{breakbox}

\begin{breakbox}
\boxtitle{db4o Native Query:}
\java{java_code/native_query.java}
\end{breakbox}

\begin{breakbox}
\boxtitle{db4o SODA Query:}
\java{java_code/soda_query.java}
The \lstinline[language=Java]{.not()} could also be:
\begin{itemize}
	\item \lstinline[language=Java]{.and(Constraint);}
	\item \lstinline[language=Java]{.or(Constraint);}
	\item \lstinline[language=Java]{.greater();}
\end{itemize}
\end{breakbox}























